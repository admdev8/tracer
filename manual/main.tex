\documentclass[11pt,a4paper,oneside]{book}

\usepackage{cmap}

\ifdefined\RUSSIAN
\usepackage[english,russian]{babel}
\usepackage[T2A]{fontenc}
%\usepackage{paratype}
%\renewcommand*\familydefault{\sfdefault}
% http://www.emerson.emory.edu/services/latex/latex_169.html
%\newcommand{\lstlistingsize}{\scriptsize}
\else
\usepackage[english]{babel}
%\usepackage[russian,english]{babel}
%\usepackage[T2A]{fontenc}
\usepackage[T1]{fontenc}
\usepackage[default]{sourcesanspro}
%\newcommand{\lstlistingsize}{\footnotesize}
\fi

\usepackage[utf8]{inputenc}
\usepackage{comment}
\usepackage{listings}
\usepackage{ulem}
\usepackage{url}
\usepackage{graphicx}
\usepackage{listingsutf8}
\usepackage[cm]{fullpage}
\usepackage{color}
\usepackage{fancyvrb}
\usepackage{xspace}
\usepackage{framed}
\usepackage{ccicons}
\usepackage{amsmath}
\usepackage[table]{xcolor}% http://ctan.org/pkg/xcolor
\usepackage[]{hyperref} % should be last

\definecolor{lstbgcolor}{rgb}{0.94,0.94,0.94}
\newcommand{\TT}[1]{\texttt{#1}}
\newcommand{\IT}[1]{\textit{#1}}
\newcommand{\IFRU}[2]{\iflanguage{russian}{#1}{#2}}
\newcommand*{\EN}[1]{\iflanguage{english}{#1}}
\newcommand*{\RU}[1]{\iflanguage{english}{}{#1}}
\newcommand{\forexample}{\IFRU{Например}{For example}:}
	

\newcommand{\TITLE}{\IFRU{Tracer: руководство пользователя}
{Tracer: users' manual}}
\newcommand{\AUTHOR}{\IFRU{Денис Юричев}{Dennis Yurichev}}
\newcommand{\EMAIL}{dennis@yurichev.com}

\hypersetup{
    pdftex,
    colorlinks=true,
    allcolors=blue,
    pdfauthor={\AUTHOR},
    pdftitle={\TITLE}
    }

\selectlanguage{english}

\lstset{
    backgroundcolor=\color{lstbgcolor},
%    basicstyle=\ttfamily\footnotesize,
    basicstyle=\ttfamily,
    breaklines=true,
    frame=single,
    inputencoding=cp1251,
    columns=fullflexible,keepspaces,
}

\begin{document}

\VerbatimFootnotes

\frontmatter

\begin{titlepage}
\begin{center}
\vspace*{\fill}
\LARGE \TITLE

\vspace*{\fill}

\large \AUTHOR

\large \TT{<\EMAIL>}
\vspace*{\fill}
\vfill

\ccbyncnd

\textcopyright 2013, \AUTHOR. 

\IFRU{Это произведение доступно по лицензии Creative Commons «Attribution-NonCommercial-NoDerivs» 
(«Атрибуция — Некоммерческое использование — Без производных произведений») 3.0 Непортированная. 
Чтобы увидеть копию этой лицензии, посетите}
{This work is licensed under the Creative Commons Attribution-NonCommercial-NoDerivs 3.0 Unported License. 
To view a copy of this license, visit} \url{http://creativecommons.org/licenses/by-nc-nd/3.0/}.

\IFRU{Дата компиляции этой PDF}{This PDF compilation date}: {\large \today}.

\IFRU{Англоязычная версия текста (а также сам tracer) также доступна по ссылке \url{http://yurichev.com/tracer-ru.html}}
{Russian language version of this text (as well as tracer itself) is also accessible at \url{http://yurichev.com/tracer-en.html}}
\end{center}
\end{titlepage}

\tableofcontents
\cleardoublepage

\include{preface}

\chapter{\IFRU{Посвящение}{Homage}}

\IFRU{Опции \TT{BPX}, \TT{BPMB/BPMW/BPMD} названы так же как и в SoftICE, великолепном отладчике прошлого.}
{\TT{BPX}, \TT{BPMB/BPMW/BPMD} options are named after those present in SoftICE, excellent debugger of the past.}

\chapter{\IFRU{Благодарности}{Thanks}}

Alex Ionescu.

\mainmatter

\chapter{\IFRU{Общие опции}{General options}}

\verb|-l:<fname.exe>|: \IFRU{загрузка процесса}{load process}.

\verb|-c:<cmd_line>|: \IFRU{задание командной строки для загружаемого процесса}{define command line for loading process}.

\forexample{}

\begin{lstlisting}
tracer.exe -l:bzip2.exe -c:--help
\end{lstlisting}

\IFRU{Если командная строка содержит пробелы}{If command line contain spaces:}:

\begin{lstlisting}
tracer.exe -l:rar.exe "-c:a archive.rar *"
\end{lstlisting}

\verb|-a:<fname.exe or PID>|: \IFRU{присоединение к запущенному процессу по его имени или PID}{attach to running process by file name or PID number}.

\IFRU{Процесс с таким именем должен быть загружен. Если процессов с таким имененем загружено несколько, tracer присоеденяется ко всем сразу одновременно.}{Process with that filename should be already loaded. If there're several processes with the same name, tracer will attach to all of them simultaneously.}

\TT{-{}-loading}: \IFRU{вывод имен файлов и базовых адресов для всех загружаемых модулей (обычно, это DLL-файлы).}{dump all module filenames and base addresses while loading (it's DLL files often).}

\TT{-{}-child}: \IFRU{присоеденяться в том числе и к процессам порождаемых главным процессом}{attach to all child processes too}.

\IFRU{Например, вы можете запустить \TT{tracer.exe -{}-child -l:cmd.exe}, откроется консольное окно cmd.exe и tracer будет присоеденятся к каждому процессу запущенному внутри командного интерпретатора.}{For example, you could run \TT{tracer.exe -{}-child -l:cmd.exe}, this will open console cmd.exe window and every process running inside command interpreter will be handled by tracer.}

\TT{-{}-allsymbols[:<regexp>]}: \IFRU{вывод всех символов в процессе загрузки по регулярному выражению}{dump all symbols during load or by regular expression}:

\TT{-{}-allsymbols:somedll.dll!.*} \IFRU{опция может быть использована для вывода всех символов в некоторой DLL}{can be used for dumping all symbols in some DLL}.

\TT{-{}-allsymbols:.*printf} \IFRU{выведет что-то вроде}{will print something like this}:

\begin{lstlisting}
New symbol. Module=[ntdll.dll], address=[0x77C004BC], name=[_snprintf]
New symbol. Module=[ntdll.dll], address=[0x77B8E61F], name=[_snwprintf]
...
New symbol. Module=[msvcrt.dll], address=[0x75725F37], name=[vswprintf]
New symbol. Module=[msvcrt.dll], address=[0x75726649], name=[vwprintf]
New symbol. Module=[msvcrt.dll], address=[0x756C3D68], name=[wprintf]
\end{lstlisting}

\verb|-s|: \IFRU{вывод стека вызовов перед каждым прерыванием}{dump call stack before each breakpoint}.

\forexample{}

\verb|tracer.exe -l:hello.exe -s bpf=kernel32.dll!WriteFile,args:5|

\IFRU{Мы увидим}{We will see}:

\begin{lstlisting}
23B4 (0) KERNEL32.dll!WriteFile (7, "hello to tracer!\r\n", 0x0000000E, 0x0017E3A4, 0) (called from 0x7317754E (MSVCR90.dll!_lseeki64+0x56b))
Call stack of thread 0x23B4
return address=731778D8 (MSVCR90.dll!_write+0x9f)
return address=7313FB4A (MSVCR90.dll!_fdopen+0x1c0)
return address=7313F70C (MSVCR90.dll!_flsbuf+0x6e1)
return address=73141E50 (MSVCR90.dll!printf+0x84)
return address=0040100E (hello.exe!BASE+0x100e)
return address=0040116F (hello.exe!BASE+0x116f)
return address=76FCE4A5 (KERNEL32.dll!BaseThreadInitThunk+0xe)
return address=77C9CFED (ntdll.dll!RtlCreateUserProcess+0x8c)
return address=77C9D1FF (ntdll.dll!RtlCreateProcessParameters+0x4e)
23B4 (0) KERNEL32.dll!WriteFile -> 1
\end{lstlisting}

\IFRU{Вывод стека вызова очень удобен, например, мы имеем программу показывающую окно с сообщением и перехватывая вызов \TT{USER32.DLL!MessageBoxA} мы можем увидеть путь к этому вызову.}
{Stack dump can be very useful, for example, we have a program showing Message Box once and by intercepting \TT{USER32.DLL!MessageBoxA} call we can see a path to this call.}

\IFRU{Возможность вывода стека доступна для всех типов прерываний BPF/BPX/BPM.}{Stack dump feature available for all BPF/BPX/BPM features.}

\IFRU{Замечание: эта возможность пока не очень хорошо работает в x64.}{Note: this feature doesn't working in x64 version very well (yet).}

\IFRU{Если указана опция \IT{-{}-dump-fpu}, состояние регистров FPU будут показываться.}
{If \TT{-{}-dump-fpu} option is set, FPU registers state will be dumped.}

\IFRU{Если указана опция \IT{-{}-dump-xmm}, состояние всех регистров XMM также будут выводиться, если только регистр не пуст.}
{If \TT{-{}-dump-xmm} option is set, each XMM registers state will be dumped (if needed) too, unless it is empty.}

\IFRU{Если указана опция \IT{-{}-dump-seh}, вся доступна информация о SEH будет выведена.
Для отображения информации SEH4, нужен доступ к переменной \TT{security\_cookie}, 
tracer будет искать её по имени в файле \TT{.MAP} или \TT{.PDB}.}
{If \TT{-{}-dump-seh} option is set, all SEH related information will be dumped.
For SEH4 information dumping, tracer will use \TT{security\_cookie} variable, it will search for it
by name in \TT{.MAP} or \TT{.PDB} files.}

\TT{-t}: \IFRU{записывать дату и время перед каждой строкой в лог:}{write timestamp at each log line:}

\TT{-{}-version}: \IFRU{вывести номер текущей версии и дату/время компиляции, а также проверить наличие новой версии доступной для скачивания.}{print current version and date/time of compilation, and also, check for update available for download.}

\forexample{}

\begin{lstlisting}
tracer.exe -l:bzip2.exe bpf=cygwin1.dll!fprintf,args:2 -t
\end{lstlisting}

\begin{lstlisting}
[2013-07-03 07:15:10:056] TID=13056|(0) cygwin1.dll!fprintf (0x611887b0, "%s: For help, type: `%s --help'.\n") (called from bzip2.exe!OEP+0x15f1 (0x4025f1))
[2013-07-03 07:15:10:058] TID=13056|(0) cygwin1.dll!fprintf () -> 0x27
\end{lstlisting}

\IFRU{Эта возможность полезна тогда, когда нужно записывать в журнал время каких-то событий, например, когда именно некая программа обращается к сети.}{This feature is useful when one need to log time of some events into journal, like, when exactly some program accessed network.}

\TT{-{}-help}: \IFRU{помощь.}{print help.}

\TT{-q}: \IFRU{запрет любого вывода в консоль и лог.}{be quiet, no output to console and log file.}

\TT{@}: \IFRU{опция позволяет сохранить все опции в текстовом файле и использовать их многократно:}{option can be used along with any other options:}

\begin{lstlisting}
tracer.exe @filename
\end{lstlisting}

\IFRU{Каждая строка файла представляет опцию. Это очень удобно для длинных и/или часто используемых опций, как байтмаски (смотрите ниже).}{Each line in the profile represents an option. This can be useful for lengthy and/or often used options, like bytemasks (see below).}

\IFRU{Опция \TT{@} также может использоваться с любыми другими опциями:}
{\TT{@} option can be used along with any other options:}

\begin{lstlisting}
tracer.exe -l:filename.exe @additional_options @even_more_options
\end{lstlisting}



\chapter{\IFRU{Как в tracer задается адрес}{How address is defined in tracer}}

\IFRU{Имеется три возможности задать адрес прерывания.}{There're 3 ways to define breakpoint address.}

\begin{itemize}

\item \IFRU{Используя шестнадцатиричный адрес}{By hexadecimal address}: 
\TT{0x00400000} --- \IFRU{так задается абсолютный адрес внутри win32-процесса}
{that's how address inside of win32-process is to be set}.
\IFRU{Обратите внимание, что изменение базы
загрузки PE-модуля не учитывается, так что, если, например, в IDA, или ином дизассемблере, вы видите один адрес,
то этот код все же може быть загружен по другим адресам в памяти процесса 
(вы можете использовать опцию \TT{-{}-loading}, чтобы увидеть,
по каким базовым адресам загружаются модули).}
{Please note: loading base changing of PE-module is not working out here, so,
if you see some address in IDA or any other disassembler, that piece of code may be loaded to another address in
memory process (you can use \TT{-{}-loading} options to see, on which base address modules are being loaded).}

\IFRU{А для того, чтобы указать некий адрес в определенном PE-модуле, адрес должен быть задан так}{So, to set
some arbitrary address in specific PE-module, it should be set as}: 
\TT{module.dll!0x400000} --- \IFRU{и это адрес автоматически подкорректируется, 
если модуль будет загружен по другому базовому адресу}{and this address will be corrected automatically if module
will be loaded on another base address}.

\item \IFRU{Используя символ}{By symbol}.

\forexample{} \TT{kernel32.dll!writefile}

\IFRU{Здесь можно использовать регулярные выражения.
Например: \TT{.*!printf</i>}: tracer будет искать символ \TT{printf} в каждом загружаемом модуле. Если этот символ имеется в разных модулях, tracer будет использовать только из того модуля, который был загружен раньше всех.}
{Regular expressions can be used here.
For example: \TT{.*!printf}: tracer will look for \TT{printf} symbol in each loading module. If the same name occurs in different modules, tracer will use only the first occurence.}

\IFRU{Для регулярных выражений используется синтаксис POSIX Extended Regular Expression (ERE)}{POSIX Extended Regular Expression (ERE) syntax is used here for regular expressions}.

\IFRU{Из-за того что здесь задается регулярное выражение, некоторые символы, такие как \TT{?}, \TT{.} нужно 
\IT{escape-ть}.
Например, чтобы задать адрес \TT{?method@class@@QAEHXZ}, нужно указывать \TT{\textbackslash{}?method@class@@QAEHXZ}.}
{Since this is regular expression, some symbols, like \TT{?}, \TT{.} should be \IT{escaped}. 
For example, in order to set \TT{?method@class@@QAEHXZ} address, it should be set as \TT{\textbackslash{}?method@class@@QAEHXZ}.}

\IFRU{Смещение также можно использовать. Например: \TT{file.exe!BASE+0x1234} (BASE это предопределенный символ, он равен базовому адресу PE-модуля) либо \TT{file.exe!label+0xa}.}{Offset is allowed here. For example: \TT{file.exe!BASE+0x1234} (base is predefined symbol equals to PE file base) or \TT{file.exe!label+0xa}.}

\IFRU{Символ \TT{OEP} также доступен, например}{\TT{OEP} symbol is also available, for example}: \TT{file.exe!OEP}.

\begin{comment}
\item \IFRU{Используя байтмаску}{By bytemask}.

\IFRU{Иногда нужна возможность установить прерывание внутри модуля по шаблону, например такому, какие используются в сигнатурах IDA.}{Sometimes, we would like to define breakpoint inside of an executable by byte pattern, just like IDA signature.}

\forexample{}

\begin{lstlisting}
tracer.exe -l:cycle.exe bpf=558BEC8B45085068........FF15........83C4085DC3,args:1
\end{lstlisting}

\IFRU{tracer будет искать этот шаблон во всех исполняемых секциях каждого загружаемого модуля и будет выводить:}{tracer will look for this pattern in each executable section of each loading module and will print:}

\begin{lstlisting}
===============================================================================
cycle.exe: searching for bytemask_0 in .text section...
bytemask_0 is resolved to address 0x00401000 (cycle.exe)
===============================================================================
\end{lstlisting}

\IFRU{Две точки вместо двух шестнадцатиричных цифр означают \IT{любой байт}. Это очень удобно для пропуска FIXUP-ов, итд.}{Two dots instead of two hexadecimal digits mean \IT{any byte}. This can be useful for skipping FIXUPs, etc.}

\IFRU{Возможно также указать \TT{[skip:n]} для пропуска нескольких байт внутри байтмаски, например: \TT{55[skip:10]1100} эквивалентно \TT{bpf=55....................1100}.}{It is also possible to mention \TT{[skip:n]} to skip several bytes during bytemask search, for example: \TT{55[skip:10]1100} --- this is equivalent to \TT{bpf=55....................1100}.}

\IFRU{Замечание: обычно, если байтмаской задается функция, то достаточно определить только начало функции.}{Note: if some function is defined by bytemask, only bytes from the beginning of function are enough.}

\IFRU{Замечание: байтмаска может встречаться много раз в одном и том же модуле. tracer предупредит об этом, но будет использовать только первую найденную. Это значит что байтмаска для функции должна задаваться аккуратно.}{Note: bytemask can appear more than once in one module. tracer will warn us on, and use only the first occurence by default. This means that function byte pattern is to be constructed accurately.}
\end{comment}

\end{itemize}


\chapter{\IFRU{BPF: установка прерывания на исполнение функции}{BPF: set breakpoint on function execution}}

\IFRU{Опция BPF, в каком-то смысле, похожа на работу утилиты}{BPF option, in a way, it is a kind of} strace\footnote{\url{http://en.wikipedia.org/wiki/Strace}}.

\IFRU{Главные отличия от strace:}{Significant differences with strace are:}

\begin{itemize}
\item tracer \IFRU{работает только в win32/win64.}{is win32/win64 only.}
\item \IFRU{Прерыванием может быть любоая функция а не только системные вызовы.}{Breakpoints not just system calls, but any function.}
\item \IFRU{Только 4 прерывания из-за ограничений архитектуры x86.}{Only 4 breakpoints, because of x86 architecture limitation.}
\end{itemize}

\IFRU{BPF с адресом но без дополнительных опций будет только показывать момент вызова функции и то что она возвращает.}{BPF option with address without additional options will only track the moment when function was called and what it returns.}

\forexample{}

\begin{lstlisting}
tracer.exe -l:bzip2.exe bpf=kernel32.dll!WriteFile
\end{lstlisting}

\begin{lstlisting}
1188 (0) KERNEL32.dll!WriteFile () (called from 0x610AC912 (cygwin1.dll!sigemptyset+0x1022))
1188 (0) KERNEL32.dll!WriteFile -> 1
\end{lstlisting}

\IFRU{Замечание: tracer не знает о том что функция может иметь тип \IT{void} (т.е., не возвращает ничего). Таким образом, tracer выводит просто то что находится в регистре \TT{EAX}/\TT{RAX} на момент выхода из функции.}{Note: tracer doesn't know some function is void type, e.g., it doesn't return any value. So it just takes the value at \TT{EAX}/\TT{RAX} register.}

\section{\IFRU{Опции}{Options}}

\subsection{ARGS}

\TT{ARGS:<number>}: \IFRU{определить количество агрументов для перехватываемой функции}{define arguments number for the function we would like to intercept}.

\forexample{}

\begin{lstlisting}
tracer.exe -l:bzip2.exe -c:--help bpf=kernel32.dll!WriteFile,args:5
\end{lstlisting}

\begin{lstlisting}
09D0 (0) KERNEL32.dll!WriteFile (0x0000001B, "   If no file names are given, bzip2 compresses or decompresses", 0x0000003F, "?", 0)
09D0 (0) KERNEL32.dll!WriteFile -> 1
09D0 (0) KERNEL32.dll!WriteFile (0x0000001B, "   from standard input to standard output.  You can combinesses", 0x0000003B, ";", 0)
09D0 (0) KERNEL32.dll!WriteFile -> 1
09D0 (0) KERNEL32.dll!WriteFile (0x0000001B, "   short flags, so `-v -4' means the same as -v4 or -4v, &amp;c.ses", 0x0000003C, "<", 0)
09D0 (0) KERNEL32.dll!WriteFile -> 1
\end{lstlisting}

\IFRU{То что мы видим это попытку вывести 5 агрументов функции при каждом вызове функции WriteFile().}{What we see here is an attempt to read 5 arguments at each WriteFile function call.}
\IFRU{Если аргумент является указателем в пределах памяти процесса и то на что он указывает может быть интерпретировано как ASCII-строка, она будет выведена.}{If some of these arguments are pointers to some area within process memory, and the data at the pointer can be interpreted as ASCII string, it will be printed instead.}
\IFRU{Это очень удобно для перехвата строковых функций таких как strcmp(), strlen(), strtok(), atoi(), итд.}{This is useful when intercepting string functions like strcmp(), strlen(), strtok(), atoi(), and so on.}

\IFRU{Ошибится в количестве аргументов не страшно (кроме случая использования опции \TT{skip\_stdcall}, смотрите ниже). Если указанное количество аргументов больше чем на самом деле, возможно, значения из локальных переменных вызывающей функции будут выведены. Или какой-нибудь случайный мусор. Если заданное количество аргументов меньше чем на самом деле, только часть аргументов будет выведена.}{It is not a problem to make mistake on arguments number (except using \TT{skip\_stdcall} option, see below). If defined arguments number greater than real, captured local variables of caller function probably will be printed. Or any other useless junk. If defined arguments number is less than real, then only part of arguments will be visible.}

\subsection{RT}

\TT{RT:<number>}: \IFRU{подставить другое возвращаемое значение в момент выхода из функции, на лету.}{replace the returning value of any function by something else, on fly.}

\begin{lstlisting}
tracer.exe -l:filename.exe bpf=function,args:1,rt:0x12345678
\end{lstlisting}

tracer \IFRU{запишет это значение в регистр \TT{EAX}/\TT{RAX} в момент выхода из функции.}{will put this value to \TT{EAX}/\TT{RAX} right at the moment when function exited.}

\TT{SKIP}: \IFRU{пропустить выполнение функции. Эта опция может использоваться вместе с опцией \TT{RT}.}{bypass a function. This can be used with \TT{RT} option too.}

\begin{lstlisting}
tracer.exe -l:filename.exe bpf=function,args:1,rt:0x12345678,skip
\end{lstlisting}

\IFRU{Это означает что в момент начала выполнения функции, управление сразу будет передано на выход и возвращаемое значение будет установлено в 0x12345678.}{This means that the function just gets bypassed and its return value is fixed at 0x12345678.}

\IFRU{Замечание: без префикса "0x", это значение будет интерпретироваться как десятичное число.}{Note: without "0x" prefix, this value would be interpreted as decimal number.}

\TT{SKIP\_STDCALL}: \IFRU{то же что и \TT{SKIP}, только для stdcall-функций.}{the same as <b>SKIP</b> option but rather used for stdcall functions.}

\IFRU{Разница между типами функций cdecl и stdcall в том что функция типа сdecl на выходе не выравнивает указатель стека (вызывающая функция должна сделать это). Функция типа stdcall выравнивает указатель стека. cdecl это наиболее используемый тип функций. Хотя, stdcall используется в MS Windows. Так что, если вы хотите пропустить выполнение какой-либо функции в KERNEL32.DLL или USER32.DLL, вы должны использовать \TT{skip\_stdcall}. Следовательно, в этом случае, tracer должен знать точное количество аргументов, а без этого процесс может упасть.}{The difference between cdecl and stdcall calling conventions is just that cdecl function doesn't align stack pointer at exit (caller should do this). stdcall function aligns stack pointer at exit. cdecl is the most used calling convention. However, stdcall is used in MS Windows. So, if you would like to skip a function in KERNEL32.DLL or USER32.DLL, you should use \TT{skip\_stdcall}. Consequently, in this case, tracer must know the exact arguments number, without it the process may crash.}\footnote{\IFRU{Смотрите также}{See also}: X86 calling conventions\url{http://en.wikipedia.org/wiki/X86_calling_conventions}}

\IFRU{Если вы хотите подавить все вызовы функции WriteFile:}{If you'd like to suppress all WriteFile calls, do this:}

\begin{lstlisting}
tracer.exe -l:hello.exe bpf=kernel32.dll!WriteFile,args:5,skip_stdcall,rt:1
\end{lstlisting}

\IFRU{Не забывайте возвращать 1, для того чтобы вызываемая функция не заподозрила ничего!}{Don't forget to make it return 1, so the caller will not suspect anything!}
\IFRU{Количество аргументов функции WriteFile --- 5. Поменяйте это значение на что-то другое и процесс упадет.}{WriteFile arguments number is just 5. Change it to something different, and process crashes.}

\IFRU{Замечание: тип функции stdcall отсутствует в Windows x64, так что эта опция отсутствует в 64-битной версии tracer.}{Note: stdcall calling convention is absent in Windows x64, so this option is absent in win64-version of tracer.}

\subsection{UNICODE}

\TT{UNICODE}: \IFRU{трактовать строки в аргументах как юникодные (два байта на каждый символ). Это может быть полезно для перехвата win32-функций с суффиксом W, например, MessageBoxW.}{treat strings in arguments as unicode (widechar). This could be helpful if you intercept unicode win32 functions with W suffix, for example, MessageBoxW.}

\IFRU{К сожалению, tracer умеет автоматически выявлять только строки использующие первую половину таблицы ASCII, так что строки на других языках кроме тех что используют латиницу не будут выявлены автоматически.}{Unfortunately, tracer can only automatically detect first half of ASCII table, so multilingual unicode strings will not be detected.}

\subsection{DUMP\_ARGS}

\TT{DUMP\_ARGS:<size>}: \IFRU{дампить память по аргументам функции (если она чиается) с ограничением size.}{dump memory on argument (if readable) limited by max size.}

\IFRU{Если аргумент функции содержит указатель на читаемый блок памяти, он будет выведен.}{If argument contain pointer to valid memory block, it will be printed.}

\IFRU{На момент выхода из функции, если блок в памяти изменился, то разница будет выведена также.}{At the function exit, if memory block contents was changed, difference will be printed too.}

\forexample{}

\begin{lstlisting}
tracer64.exe -l:test_getlocaltime.exe bpf=.*!getlocaltime,args:1,dump_args:0x30
\end{lstlisting}

\begin{lstlisting}
TID=6660|(0) KERNEL32.dll!GetLocalTime (0x12ff00) (called from 0x14000100f (getlocaltime.exe!BASE+0x100f))
Dump of buffer at argument 1 (starting at 1)
000000000012FF00: 28 FF 12 00 00 00 00 00-00 00 00 00 00 00 00 00 "(..............."
000000000012FF10: 01 00 00 00 00 00 00 00-73 11 00 40 01 00 00 00 "........s..@...."
000000000012FF20: 00 00 00 00 00 00 00 00-00 00 00 00 00 00 00 00 "................"
TID=6660|(0) KERNEL32.dll!GetLocalTime -> 0x150
Dump difference of buffer at argument 1 (starting at 1)
0000000000000000: D9 07 0C    06    05   -05    10    24    50 01 "... . . . . $ P."
\end{lstlisting}

\IFRU{Таким образом мы можем увидеть как win32-функция GetLocalTime() заполняет структуру SYSTEMTIME.}
{Now we can see how GetLocalTime win32 function fill SYSTEMTIME structure.}

\subsection{ARGn\_TYPE}

\TT{ARGn\_TYPE:}: \IFRU{задать тип аргумента}{set type of argument}. \IFRU{Аргументы нумеруются от}
{Arguments are numbered from} 1.
\IFRU{Из доступных типов пока только}{Types available so far only} QString.
\IFRU{Следовательно, теперь можно задать BPF-брякпоинт вроде}{Hence, it is possible to set 
BPF-breakpoint like}:

\begin{lstlisting}
bpf=filename.exe!address,args:3,arg3_type:QString
\end{lstlisting}

... \IFRU{и QString будет выведен}{and QString will be dumped}.
\IFRU{Номер аргумента не может быть больше чем задано в опции ARGS}
{Argument number cannot be greater than what was set in ARGS option}.
\RU{Опция }\TT{ARGS} \IFRU{должна быть задана перед}{option should be set before} \TT{ARGn\_TYPE:}.

% TODO: *QString

\subsection{PAUSE}

\TT{PAUSE:<number>}: \IFRU{Сделать паузу, в миллисекундах. 1000 --- одна секунда. Удобно для тестирования, для создания \IT{искусственных} задержек. К примеру, полезно знать, как поведет себя программа при очень медленной сети:}{Make a pause in milliseconds. 1000 --- one second. It is convenient for testing, for creating artifical delays. For example, it is important to know program's behaviour in very slow network environment:}

\begin{lstlisting}
tracer.exe -l:test1.exe bpf=WS2_32.dll!WSARecv,pause:1000
\end{lstlisting}

\IFRU{... или же, если будет считывать информацию с какого-то очень медленного носителя:}{... or if it will read from some very slow storage:}

\begin{lstlisting}
tracer.exe -l:test1.exe bpf=kernel32.dll!ReadFile,pause:1000
\end{lstlisting}

\subsection{RT\_PROBABILITY}

\TT{RT\_PROBABILITY:<number>}: \IFRU{Используется в паре с опцией \TT{RT:}, задает вероятность срабатывания \TT{RT}. К примеру, если был задан \TT{RT:0} и \TT{RT\_PROBABILITY:30\%}, то 0 будет подставляться вместо результата функции в 30\% случаев. Это также удобно для тестирования --- хорошо написанная программа должна корректно обрабатывать ошибки. Например, вот так мы можем симулировать ошибки выделения памяти, 1 вызов \TT{malloc()} на сотню, вернет \IT{NULL}:}{Used with \TT{RT:} option in pair, defines probability of \TT{RT} triggering. For example, if \TT{RT:0} and \TT{RT\_PROBABILITY:30\%} were set, 0 will be set instead of function's return value in 30\% of cases. It's convinient for testing --- good written program should handle errors correctly. For example, that's how we can simulate memory allocation errors, 1 \TT{malloc()} call of 100 will return \IT{NULL}:}

\begin{lstlisting}
tracer.exe -l:test1.exe bpf=msvcrt.dll!malloc,rt:0,rt_probability:1%
\end{lstlisting}

\IFRU{... в 10\% случаев, файл не будет открываться:}{... in 10\% of cases, the file will fail to open:}

\begin{lstlisting}
tracer.exe -l:test1.exe bpf=kernel32.dll!CreateFile,rt:0,rt_probability:10%
\end{lstlisting}

\IFRU{Вероятность также можно задавать и обычным образом, как число в интервале от 0 (никогда) до 1 (всегда). 10\% это 0.1, 3\% это 0.03, итд.}{Probability may be set in usual manner, as a number in 0 (never) to 1 (always) interval. 10\% is 0.1, 3\% is 0.03, etc.}

\IFRU{Об идеях, какие еще ошибки можно симулировать, читайте так же здесь}{About ideas on errors also may be simulated, read here} \href{http://blog.yurichev.com/node/43}{Oracle RDBMS internal self-testing features}.

\subsection{\IFRU{Опция TRACE}{TRACE option}}

\TT{TRACE}: \IFRU{трассировать функцию по одной инструкции и сохранять значения всех интересующих нас регистров. После исполнения, эта информация сохранится в файлы process.exe.idc, process.exe.txt, process.exe\_clear.idc. .idc-файлы являются скриптами для IDA, а к .txt файлу можно применять grep, awk, sed для поиска интересующих нас значений.}{trace each instruction in function and collect all interesting values from registers and memory. After execution, all that information is saved to process.exe.idc, process.exe.txt, process.exe\_clear.idc files. .idc-files are IDA scripts, .txt file is grepable by grep, awk and sed.}

\IFRU{Возьмем для примера функцию add\_member из статьи}{For example, let's take add\_member function from} \IT{Using Uninitialized Memory for Fun and Profit}\footnote{\url{http://research.swtch.com/2008/03/using-uninitialized-memory-for-fun-and.html}}\IFRU{}{article}:

\begin{lstlisting}
int dense[256];
int dense_next=0;
int sparse[256];

void add_member(int i)
{
	dense[dense_next]=i;
	sparse[i]=dense_next;
	dense_next++;

};

int main ()
{
	add_member(123);
	add_member(5);
	add_member(71);
	add_member(99);
}
\end{lstlisting}

\IFRU{Скомпилируем и запустим трассировку на функции add\_member (вначале узнайте адрес функции при помощи IDA):}{Let's compile it and run tracing on add\_member function (determine function address in IDA before):}

\begin{lstlisting}
tracer -l:trace_test4.exe bpf=0x00401000,trace:cc
\end{lstlisting}

\IFRU{Получим файл trace\_test4.exe.txt:}{We'll get trace\_test4.exe.txt file:}

\begin{lstlisting}
0x401000, e=       4
0x401001, e=       4
0x401003, e=       4, [0x403818]=0..3
0x401008, e=       4, [EBP+8]=5, 0x47('G'), 0x63('c'), 0x7b('{')
0x40100b, e=       4, ECX=5, 0x47('G'), 0x63('c'), 0x7b('{')
0x401012, e=       4, [EBP+8]=5, 0x47('G'), 0x63('c'), 0x7b('{')
0x401015, e=       4, [0x403818]=0..3
0x40101a, e=       4, EAX=0..3
0x401021, e=       4, [0x403818]=0..3
0x401027, e=       4, ECX=0..3
0x40102a, e=       4, ECX=1..4
0x401030, e=       4
0x401031, e=       4, EAX=0..3
\end{lstlisting}

\IFRU{Поле \IT{e} - это сколько раз была исполнена эта инструкция.}{\IT{e} field is how many times was executed this instruction.}

\IFRU{Загрузим trace\_test4.exe.idc в IDA и увидим:}{Let's execute trace\_test4.exe.idc script in IDA and we'll see:}

\begin{figure}[ht!]
\centering
\includegraphics[scale=0.66]{trace_test4.png}
\caption{trace\_test4.png}
\end{figure}

\IFRU{Понимать работу функции во время исполнения, таким образом, становится намного проще.}{Now it is much simpler to understand how this function work during execution.}

\IFRU{Исполненные инструкции подсвечиваются голубым цветом. Неисполненные остаются белыми.}{Executed instructions are highlighed by blue color. Not-executed instructions are leaved white.}

\IFRU{Чтобы стереть все комментарии и подсветку, нужно исполнить скрипт trace\_test4.exe\_clear.idc}{If you need to clear all comments and highlight, execute trace\_test4.exe\_clear.idc script.}

\IFRU{Информация в IDA-скрипте может приводится в сокращенной форме из-за того что IDA имеет ограничение на длину комментария, например: \TT{EAX=[ 64 unique items. min=0xbca6eb7, max=0xffffffed ]}. В текстовом же файле сохраняется всё, поэтому иногда этот файл может оказаться в итоге очень большим.}{All collected information in IDA-script may be reduced to shorten form like \IT{EAX=[ 64 unique items. min=0xbca6eb7, max=0xffffffed ]} (because IDA has comment size limitation). On contrary, everything is saved to text file without shortening, that is why resulting text file may be sometimes pretty big.}

\IFRU{Недостаток опции TRACE в том что она работает медленно, хотя и функции в системных DLL пропускаются (системной считается та DLL которая находится внутри \%SystemRoot\%) Вторая проблема в том что пока что не очень корректно трассируются вещи вроде исключений, setjmp/longjmp и подобных непредвиденных изменений пути исполнения кода.}{One problem of TRACE feature that it is slow, however, functions from system DLLs are skipped (system DLL is that DLL residing in \%SystemRoot\%) Another problem is that things like exceptions, setjmp/longjmp and other unexpected codeflow alterations are not correctly handled so far.}



\section{\IFRU{Примеры}{Examples}}

\subsection{\IFRU{Простое использование}{Simple usage}}

\begin{lstlisting}
tracer.exe -l:bzip2.exe bpf=.*!fprintf,args:3
\end{lstlisting}

\begin{lstlisting}
TID=5128|(0) cygwin1.dll!fprintf (0x61103150, "%s: I won't write compressed data to a terminal.\n", "bzip2") (called from 0x401e03 (bzip2.exe!BASE+0x1e03))
TID=5128|(0) cygwin1.dll!fprintf -> 0x34
TID=5128|(0) cygwin1.dll!fprintf (0x61103150, "%s: For help, type: `%s --help'.\n", "bzip2") (called from 0x401c66 (bzip2.exe!BASE+0x1c66))
TID=5128|(0) cygwin1.dll!fprintf -> 0x27
\end{lstlisting}

\subsection{\IFRU{Перехват некоторых Windows-функций для работы с реестром}{Intercept some Windows registry access functions}}

\begin{lstlisting}
tracer.exe -l:someprocess.exe bpf=advapi32.dll!RegOpenKeyExA,args:5 bpf=advapi32.dll!RegQueryValueExA,args:6 bpf=advapi32.dll!RegSetValueExA,args:6
\end{lstlisting}

.. \IFRU{или измените суффиксы функция на W и добавьте опцию UNICODE}{or change function suffixes to W and add UNICODE option}:

\begin{lstlisting}
tracer64.exe -l:far.exe bpf=advapi32.dll!RegOpenKeyExW,args:5,unicode bpf=advapi32.dll!RegQueryValueExW,args:6,unicode bpf=advapi32.dll!RegSetValueExW,args:6,unicode
\end{lstlisting}

\subsection{\IFRU{Подавить шумный сигнал}{Suppress noisy beeping}}

\begin{lstlisting}
tracer.exe -l:beeper.exe bpf=kernel32.dll!Beep,args:2,skip_stdcall,rt:1
\end{lstlisting}

\subsection{\IFRU{Подавить диалоговое окно с сообщением}{Suppress Message Box}}

... \IFRU{и сделать так что вызываемая функция будет считать что пользователь каждый раз нажимает OK (константа IDOK равняется 1)}{by making it appear to a caller that the user presses OK every time (IDOK constant is 1)}:

\begin{lstlisting}
tracer.exe -l:filename.exe bpf=user32.dll!MessageBoxA,args:4,skip_stdcall,rt:1
\end{lstlisting}

... \IFRU{или CANCEL (константа IDCANCEL равняется 2)}{or CANCEL (IDCANCEL constant is 2)}:

\begin{lstlisting}
tracer.exe -l:filename.exe bpf=user32.dll!MessageBoxA,args:4,skip_stdcall,rt:2
\end{lstlisting}

\subsection{\IFRU{Перехват вызовов rand()}{Intercepting rand() call}}

\IFRU{Бывает весело перехватывать вызовы функции rand() в различных играх. Например, пасьянс Solitaire в Windows использует его для того чтобы сгенерировать случайный расклад. Мы можем установить возвращаемое значение rand() в ноль, и тогда Solitaire будет раздавать один и тот же расклад, всегда:}{Another fun is intercepting rand() function in various games. For example, Windows Solitaire card game use it to generate random deal. We can fix rand() return at zero, and Solitaire will do the same deal each time, forever:}

\IFRU{В}{In} Windows XP x86/x64:

\begin{lstlisting}
tracer.exe/tracer64.exe -l:c:\windows\system32\sol.exe bpf=.*!rand,rt:0
\end{lstlisting}

\IFRU{В}{In} Windows 7 x64:

\begin{lstlisting}
tracer64.exe -l:[full path to]\Solitaire.exe bpf=.*!rand,rt:0
\end{lstlisting}

\subsection{FreeCell}

\IFRU{Когда вы запускаете FreeCell в Windows (XP SP3) и нажимаете F2 (Новая игра), вы видите сообщение "Do you want to resign this game?" Мы можем подавить звуковой сигнал и сделать так что FreeCall будет думать что пользователь всегда нажимает YES:}{When you run Windows (XP SP3) FreeCell and press F2 (New game), you will get a message box "Do you want to resign this game?" We can suppress all that beeping and also make illusion to FreeCell user always press YES:}

\IFRU{Константа IDYES - 6. FreeCell использует функцию MessageBoxW - суффикс W означает уникодную версию функции MessageBox.}{IDYES constant is 6. FreeCell use MessageBoxW - W mean unicode version of MessageBox.}

\IFRU{В}{In} Windows XP SP3 x86:

%*
\begin{lstlisting}
tracer.exe -l:c:\windows\system32\freecell.exe bpf=user32.dll!messagebeep,args:1,skip_stdcall bpf=user32.dll!messageboxw,args:4,unicode,skip_stdcall,rt:6
\end{lstlisting}

%*
\begin{lstlisting}
(0) user32.dll!messagebeep (0x20) (called from freecell.exe!BASE+0x1f52 (0x1001f52))
(0) Skipping execution of this function
(0) user32.dll!messagebeep () -> 0x8
(1) user32.dll!messageboxw (0x160152, "Do you want to resign this game?", "FreeCell", 0x24) (called from freecell.exe!BASE+0x1f5f (0x1001f5f))
(1) Skipping execution of this function
(1) user32.dll!messageboxw () -> 0x8
(1) Modifying EAX register to 0x6
\end{lstlisting}

\IFRU{В}{In} Windows XP SP2 x64 Russian:

%*
\begin{lstlisting}
tracer64.exe -l:c:\windows\system32\freecell.exe bpf=user32.dll!messagebeep,args:1,skip bpf=user32.dll!messageboxw,args:4,unicode,skip,rt:6
\end{lstlisting}

%*
\begin{lstlisting}
TID=2836|(0) user32.dll!messagebeep (0x20) (called from freecell.exe!BASE+0x23f9 (0x1000023f9))
(0) Skipping execution of this function
TID=2836|(0) user32.dll!messagebeep () -> 0x8
TID=2836|(1) user32.dll!messageboxw (0x5010e, "Do you want to resign this game?", "FreeCell", 0x24) (called from freecell.exe!BASE+0x2416 (0x100002416))
(1) Skipping execution of this function
TID=2836|(1) user32.dll!messageboxw () -> 0x8
TID=2836|(1) Modifying RAX register to 0x6
\end{lstlisting}

\subsection{\IFRU{Проверка ивентов и запись в лог в Oracle RDBMS}{Oracle RDBMS Events checking and log writes}}

\IFRU{В}{In} Oracle 10.2.0.1 win64:

\begin{lstlisting}
tracer64.exe -a:oracle.exe bpf=oracle.exe!ksdpec,args:1 bpf=oracle.exe!ss_wrtf,args:3
\end{lstlisting}

( \IFRU{Смотрите также}{See also}: \url{http://blog.yurichev.com/node/14} )

\begin{lstlisting}
TID=3032|(0) oracle.exe!ksdpec (0x2743) (called from 0x9580a9 (oracle.exe!opiodr+0x105))
TID=3032|(0) oracle.exe!ksdpec -> 0xff
TID=3032|(1) oracle.exe!ss_wrtf (0x4a0, "*** 2009-12-04 06:19:01.005\n", 0x1b) (called from 0x45318d (oracle.exe!sdpri+0x22d))
TID=3032|(1) oracle.exe!ss_wrtf -> 1
TID=3032|(1) oracle.exe!ss_wrtf (0x4a0, "OPI CALL: type=107 argc= 3 cursor=  0 name=SES OPS (80)\n", 0x37) (called from 0x45318d (oracle.exe!sdpri+0x22d))
TID=3032|(1) oracle.exe!ss_wrtf -> 1
TID=3032|(0) oracle.exe!ksdpec (0x2743) (called from 0x9580a9 (oracle.exe!opiodr+0x105))
TID=3032|(0) oracle.exe!ksdpec -> 0xff
TID=3032|(1) oracle.exe!ss_wrtf (0x4a0, "OPI CALL: type=59 argc= 4 cursor=  0 name=VERSION2\n", 0x32) (called from 0x45318d (oracle.exe!sdpri+0x22d))
TID=3032|(1) oracle.exe!ss_wrtf -> 1
TID=3032|(0) oracle.exe!ksdpec (0x273e) (called from 0x4a00cc (oracle.exe!kslwte_tm+0x7a8))
TID=3032|(0) oracle.exe!ksdpec -> 0
TID=3032|(0) oracle.exe!ksdpec (0x273e) (called from 0x4a00cc (oracle.exe!kslwte_tm+0x7a8))
TID=3032|(0) oracle.exe!ksdpec -> 0
TID=3032|(0) oracle.exe!ksdpec (0x2743) (called from 0x9580a9 (oracle.exe!opiodr+0x105))
TID=3032|(0) oracle.exe!ksdpec -> 0xff
TID=3032|(1) oracle.exe!ss_wrtf (0x4a0, "OPI CALL: type=104 argc=12 cursor=  0 name=Transaction Commit/Rollback\n", 0x46) (called from 0x45318d (oracle.exe!sdpri+0x22d))
TID=3032|(1) oracle.exe!ss_wrtf -> 1
\end{lstlisting}

\subsection{\IFRU{Слежение за выделением памяти в}{Trace memory allocations in} Oracle 11.1.0.6.0 win32/win64}

\begin{lstlisting}
tracer.exe/tracer64.exe -a:oracle.exe bpf=.*!kghalf,args:6 bpf=.*!kghfrf,args:4
\end{lstlisting}

\begin{lstlisting}
TID=1600|(0) oracle.exe!kghalf (0x6d35af0, 0xb507ef8, 0x1000, 0, 0, "kzsrcrdi") (called from 0x1c7aa83 (oracle.exe!kzctxhugi+0x71))
TID=1600|(0) oracle.exe!kghalf -> 0xfa3ea58

TID=1600|(0) oracle.exe!kghalf (0x6d35af0, 0xb507ef8, 0x58, 1, 0x6d35530, "UPI heap") (called from 0x1e7f8b7 (oracle.exe!__PGOSF266_kwqmahal+0x5b))
TID=1600|(0) oracle.exe!kghalf -> 0xfa4d0d8

TID=1188|(0) oracle.exe!kghalf (0xda39540, 0xda39240, 0x88, 0, "ksirmdt array", 0xda39240) (called from 0x6afb5b (oracle.exe!ksz_nfy_ipga+0xf1))
TID=1188|(0) oracle.exe!kghalf -> 0x105d0b10

TID=1188|(0) oracle.exe!kghalf (0xda39540, 0xda39240, 0x48, 1, 0x1204e400, "local") (called from 0x3684a64 (oracle.exe!kjztcxini+0x58))
TID=1188|(0) oracle.exe!kghalf -> 0x105d0ab0
\end{lstlisting}

\subsection{\IFRU{Слежение за разбором SQL-выражений в}{SQL statements parsing in} Oracle RDBMS}

\IFRU{В}{In} Oracle 11.1.0.6.0 win32/win64:

\begin{lstlisting}
tracer.exe/tracer64.exe -a:oracle.exe bpf=oracle.exe!_?rpisplu,args:8 bpf=oracle.exe!_?kprbprs,args:7 bpf=oracle.exe!_?opiprs,args:6 bpf=oraclient11.dll!OCIStmtPrepare,args:6</i></p>
\end{lstlisting}

\IFRU{Замечание: регулярное выражение \IT{\_?function} покрывает оба имени: \TT{function} и \TT{\_function}.}{Note: regular expression \TT{\_?function} cover both \TT{function} and \TT{\_function}.}

\begin{lstlisting}
TID=1140|(2) oracle.exe!opiprs (0x13f029d0, "select 1 from obj$ where name='DBA_QUEUE_SCHEDULES'", 0x34, 0x10ae7f50, 0x840082, 0xd9f7a10) (called from 0x6ba3bf (oracle.exe!__PGOSF423_kksParseChildCursor+0x2dd))
TID=1140|(2) oracle.exe!opiprs -> 0
TID=1140|(2) oracle.exe!opiprs (0x13f029d0, "select 1 from sys.aq$_subscriber_table where rownum < 2 and subscriber_id <> 0 and table_objno <> 0", 0x64, 0x10ad5de8, 0, 0x13f007e0) (called from 0x6ba3bf (oracle.exe!__PGOSF423_kksParseChildCursor+0x2dd))
TID=1140|(2) oracle.exe!opiprs -> 0
TID=1140|(0) oracle.exe!rpisplu (3, 0, 0, 0, 0, 0x14430ac0, 0, 0) (called from 0x250b33c (oracle.exe!kqdGetCursor+0x106))
TID=1140|(0) oracle.exe!rpisplu -> 0
TID=1288|(2) oracle.exe!opiprs (0x17df8130, "select * from v$version", 0x18, 0x10adee60, 0, 0) (called from 0x6ba3bf (oracle.exe!__PGOSF423_kksParseChildCursor+0x2dd))
TID=1288|(1) oracle.exe!kprbprs (0xa82bc50, 0, "select timestamp, flags from fixed_obj$ where obj#=:1", 0x35, 0xffffe3e0, 0x2040800, 1) (called from 0x2ba1b1f (oracle.exe!kqldtstr+0x151))
TID=1288|(1) oracle.exe!kprbprs -> 0
TID=1288|(0) oracle.exe!rpisplu (0x1f, 0, 0, 0, 0, 0x2bb5e04, "select  BANNER from GV$VERSION where inst_id = USERENV('Instance')", 0xffffc085) (called from 0x2bbcabf (oracle.exe!kqldFixedTableLoadCols+0x157))
TID=1288|(1) oracle.exe!kprbprs (0x1090c108, 0, "select timestamp, flags from fixed_obj$ where obj#=:1", 0x35, 0xffffe3e0, 0x2040800, 1) (called from 0x2ba1b1f (oracle.exe!kqldtstr+0x151))
TID=1288|(1) oracle.exe!kprbprs -> 0
TID=1288|(1) oracle.exe!kprbprs (0x10908060, 0, "select timestamp, flags from fixed_obj$ where obj#=:1", 0x35, 0xffffe3e0, 0x2040800, 1) (called from 0x2ba1b1f (oracle.exe!kqldtstr+0x151))
TID=1288|(1) oracle.exe!kprbprs -> 0
TID=1288|(2) oracle.exe!opiprs -> 0
TID=1288|(0) oracle.exe!rpisplu -> 0
TID=1288|(0) oracle.exe!rpisplu (0x16, 0, 0, 0, 0, 0x10b3ce50, 0, 0) (called from 0x250b33c (oracle.exe!kqdGetCursor+0x106))
TID=1288|(0) oracle.exe!rpisplu -> 0
\end{lstlisting}

\subsection{\IFRU{Игнорирование неподписанных драйверов}{Ignore unsigned drivers}}

\begin{lstlisting}
tracer.exe -l:target.exe bpf=Wintrust.dll!WinVerifyTrust,rt:0
\end{lstlisting}

\subsection{\IFRU{Вывод памяти по аргументам функций}{Dump function arguments}}

\begin{lstlisting}
tracer.exe -l:rar.exe "-c:a archive.rar *.exe" bpf=kernel32.dll!writefile,args:5,dump_args:0x10
\end{lstlisting}

\IFRU{RAR записывает свою сигнатуру в начало файла archive.rar:}{RAR writting its signature to the beginning of archive.rar file:}

\begin{lstlisting}
TID=7000|(0) KERNEL32.dll!WriteFile (0x118, 0x152410, 7, 0x150fc0, 0) (called from 0x403721 (rar.exe!__GetExceptDLLinfo+0x26c8))
Dump of buffer at argument 2 (starting at 1)
00152410: 52 61 72 21 1A 07 00 00-50 30 15 00 5D 83 40 00 "Rar!....P0..].@."
Dump of buffer at argument 4 (starting at 1)
00150FC0: 00 00 00 00 21 7B 40 00-10 24 15 00 18 24 15 00 "....!{@..$...$.."
TID=7000|(0) KERNEL32.dll!WriteFile -> 1
\end{lstlisting}

\subsection{\IFRU{Вывод памяти по аргументам функций и слежение за её изменением}{Dump function arguments and track difference occured in buffers}}

\begin{lstlisting}
tracer.exe -l:rar.exe "-c:x archive.rar" bpf=kernel32.dll!readfile,args:4,dump_args:0x10
\end{lstlisting}

\IFRU{Архиватор RAR открывает файл archive.rar и первым делом читает сигнатуру:}{RAR archiver open archive.rar and read signature for the first:}

\begin{lstlisting}
TID=6148|(0) KERNEL32.dll!ReadFile (0x120, 0x17b3f8, 7, 0x174c50) (called from 0x403966 (rar.exe!__GetExceptDLLinfo+0x290d))
Dump of buffer at argument 2 (starting at 1)
0017B3F8: 00 00 00 00 00 00 00 00-00 00 00 00 48 00 00 00 "............H..."
Dump of buffer at argument 4 (starting at 1)
00174C50: 07 00 00 00 78 4C 17 00-7A 38 40 00 8C 6D 17 00 "....xL..z8@..m.."
TID=6148|(0) KERNEL32.dll!ReadFile -> 1
Dump difference of buffer at argument 2 (starting at 1)
00000000: 52 61 72 21 1A 07      -                        "Rar!..          "
\end{lstlisting}

\subsection{\IFRU{Примеры опции TRACE}{TRACE feature examples}}

\subsubsection{\IFRU{Трассировка строковых функций}{Tracing string functions}}

\IFRU{Возьмем пример применения strtok():}{Let's take strtok() example:}

\begin{lstlisting}
// example from http://www.cplusplus.com/reference/clibrary/cstring/strtok/

/* strtok example */
#include <stdio.h>
#include <string.h>

int main ()
{
  char str[] ="- This, a sample string.";
  char * pch;
  printf ("Splitting string \"%s\" into tokens:\n",str);
  pch = strtok (str," ,.-");
  while (pch != NULL)
  {
    printf ("%s\n",pch);
    pch = strtok (NULL, " ,.-");
  }
  return 0;
}
\end{lstlisting}

\IFRU{И трассируем функцию main():}{Let's trace main() function:}

\begin{lstlisting}
tracer.exe -l:trace_test1.exe bpf=0x00401000,trace:cc
\end{lstlisting}

\IFRU{После исполнения скрипта в IDA (показана только тело цикла \IT{while}):}{After executing resulting .idc script in IDA (only \IT{while} loop body showed here):}

\begin{figure}[ht!]
\centering
\includegraphics[scale=0.66]{trace_test1.png}
\caption{trace\_test1.png}
\end{figure}

\IFRU{Замечание: "a" это слишком короткая строка для автоматического детектора строк в tracer, поэтому её здесь нет, вместо нее адрес этой строки.}{Note: "a" is too short string for automatic string detector in tracer, that is why it is absent and its address here instead.}

\subsubsection{\IFRU{Трассируем quicksort()}{Let's trace quicksort()}}

\IFRU{Возьмем известный пример:}{Use well-known example:}

\begin{lstlisting}
//http://cplus.about.com/od/learningc/ss/pointers2_8.htm

/* ex3 Sorting ints with qsort */
//

#include <stdio.h>
#include <stdlib.h>

int comp(const int * a,const int * b) 
{
  if (*a==*b)
    return 0;
  else
    if (*a < *b)
        return -1;
     else
      return 1;
}

int main(int argc, char* argv[])
{
   int numbers[10]={1892,45,200,-98,4087,5,-12345,1087,88,-100000};
   int i;

  /* Sort the array */
  qsort(numbers,10,sizeof(int),comp);
  for (i=0;i<9;i++)
    printf("Number = %d\n",numbers[ i ]);
  return 0;
}
\end{lstlisting}

\IFRU{Трассируем функцию comp():}{Let's trace comp() function:}

\begin{lstlisting}
tracer.exe -l:trace_test2.exe bpf=0x00401030,trace:cc
\end{lstlisting}

\IFRU{Получим после исполнения скрипта в IDA:}{We will get after .idc script execution in IDA:}

\begin{figure}[ht!]
\centering
\includegraphics[scale=0.66]{trace_test2.png}
\caption{trace\_test2.png}
\end{figure}

\IFRU{В примере все значения уникальны, одинаковых нет. Таким образом, нет ситуации когда comp() возвращает ноль. Поэтому здесь мы видим что часть comp() возвращающая ноль (xor eax,eax / retn) не была исполнена ни разу.}{In this example all values are unique, there are no equal ones. Therefore, there are no situation when comp() function returning zero. That is why we see that the comp() part returning zero (xor eax,eax / retn) was not executed.}







\include{BPX}

\include{BPM}

\include{one-time-INT3}

\chapter{\IFRU{Общее для BPX и BPF}{Things shared in BPX and BPF}}

\IFRU{Во время дампа памяти, этот участок анализируется на наличие в нем известных символов, и они
выводятся, если это возможно}{While memory dumping, the memory region to be dumped is analysed 
for known symbols inside, and they printed if possible}.

\IFRU{Это может быть крайне полезно если некая структура имеет указатели на функции}
{This may be very useful if some structure contain pointers to functions}.

\IFRU{Пример}{Example}:

\begin{lstlisting}
#include <stdio.h>

int global_var_1;
int global_var_2;
int global_var_3;

void f(void** a)
{
	a[3]=&global_var_3;
};

int main()
{
	void* a[5];
	int i;

	a[0]=a[1]=a[2]=a[3]=a[4]=NULL;
	a[1]=&global_var_1;
	a[3]=&global_var_2;
	f(a);
	for (i=0; i<5; i++)
		printf ("0x%p\n", a[i]);
};
\end{lstlisting}

\begin{lstlisting}
tracer.exe -l:1.exe bpf=1.exe!f,args:1,dump_args:0x10
\end{lstlisting}

\begin{lstlisting}
(0) 1.exe!f(0x3ffa74) (called from 1.exe!func3+0x2c (0x2519dc))
Argument 1/1 
003FFA74: 00 00 00 00 80 54 25 00-00 00 00 00 84 54 25 00 ".....T%......T%."
Argument 1/1 +0x4: 1.exe!global_var_1
Argument 1/1 +0xC: 1.exe!global_var_2
(0) 1.exe!f() -> 0x3ffa74
Argument 1/1 difference
00000000:                        -            7C          "            |   "
Argument 1/1 before +0xC: 1.exe!global_var_2
Argument 1/1 after +0xC: 1.exe!global_var_3
\end{lstlisting}

\chapter{\IFRU{Взаимодействие во время работы}{Interacting while running}}

1) \IFRU{Нажмите ESC или Ctrl-C для отсоединения от запущенного процесса.}{Press ESC or Ctrl-C to detach from the running process.}

2) \IFRU{Нажмите пробел чтобы увидеть стеки вызовов для каждого треда и процесса.}{Press SPACE to see current call stacks for each thread of each process.}

\IFRU{Например: присоеденитесь к какой-нибудь запущенной программе с открытым окном с сообщением (Message Box), нажмите пробел и возможно вы увидите, что привело к появлению этого окна.}{For example: attach to some running application with opened Message Box, press SPACE and see what probably caused it.}

\IFRU{Замечание: вывод стека вызовов пока плохо работает в tracer64.}{Note: dump call stack feature is not very well working in tracer64.}

\chapter{\IFRU{Отсоединение от процесса}{Detaching}}

\IFRU{tracer использует функцию DebugActiveProcessStop() для отсоединения от запущенного процесса. Эта функция присутствует во всех современных ОС базирующихся на NT, возможно, кроме Windows NT и Windows 2000. Так что всё что tracer может сделать в этих ОС это убить процесс --- извините!}{tracer uses DebugActiveProcessStop() function to detach from the running process. It is present in all modern NT-based operation systems, probably, except Windows NT and Windows 2000. So all tracer can do is just to kill the running process --- sorry!}

\chapter{\IFRU{Некоторые технические заметки}{Some other technical notes}}

\IFRU{Архитектура x86 позволяет использовать до четырех прерываний одновременно. Таким образом, опции \TT{BPF/BPX/BPM} могут комбинироваться в любом порядке до четырех раз.}{x86 architecture allow to use up to 4 breakpoints simultaneously. So, \TT{BPF/BPX/BPM} features can be combined in any order up to 4 times.}

\IFRU{Возможность вывода стека вызовов предпологает что фреймы в стеке "разделены" указателем в регистре EBP:}{Stack dumping feature consider stack frames "divided" with EBP base pointer:}

\IFRU{Смотрите также}{See also}: \href{http://en.wikibooks.org/wiki/X86_Disassembly/Functions_and_Stack_Frames}{Functions and Stack Frames}

\IFRU{Это означает что любая функция которая не использует эту схему, будет исключена их стека вызовов --- непреднамеренно.}{This means that any function which doesnt use this scheme will be excluded from stack dump --- unintentionally.}

\IFRU{Замечание: в tracer64 эта возможность работает не очень хорошо.}
{Note: this feature is not performing very well in tracer64.}

\IFRU{Вся информация выводится в stdout а также записывается в файл tracer.log. Файл создается снова при каждом запуске.}{All information dumped to stdout is also written to tracer.log file. This file is created at each start.}

\IFRU{При загрузке или присоеденению к процессу, tracer проверяет все модули: главный исполняемый файл и все файлы DLL загружаемые после. Он извлекает все символы из модуля включая эксплорты DLL. 
Он также ищет файл FileName.MAP и пытается его парсить. Файл MAP имеет такой же формат как то что делает IDA. 
tracer также ищет файл FileName.SYM и пытается загрузить символы из него, трактуя его как файл символов из Oracle RDBMS: переменная окружения ORACLE\_HOME должна быть установлена для этого.
tracer также ищет файл FileName.PDB (компилируйте вашу программу в MSVC с ключом \TT{/Zi} и вы получите отладочный 
файл PDB для нее).}
{While loading or attaching, tracer will inspect all modules: main executable and all DLL files loaded after.
It will fetch all present symbols, incuding export entries of DLL files.
It will also look for FileName.MAP file and try to parse information from it.
MAP file has the same format as that produced by IDA disassembler.
tracer will also look for FileName.SYM file and try to load symbols from it, treating those as Oracle RDBMS SYM file format: ORACLE\_HOME environment value should be set for this.
tracer will also look for FileName.PDB file (compile your program in MSVC with \TT{/Zi} option 
and get debug PDB file for it).}

\IFRU{Если DLL содержит экспорты только по ординалам, т.е., без имен (например, DLL-файлы MFC), имя символа будет получено из ординала в таком формате: \TT{ordinal\_<number>}, например, \TT{ordinal\_12}.}{If DLL contain only exports by ordinals, e.g., without names (MFC DLLs, for example), the name of ordinal will be generated in compliance with \TT{ordinal\_<number>} format, for example, \TT{ordinal\_12}.}

\chapter{\IFRU{Известные проблемы}{Known issues}}

\section{Windows 2000}

\IFRU{Для работы под Windows 2000, библиотека Octothorpe должна быть скомпилирована с флагом}
{For running in Windows 2000, Octothorpe library should be compiled with this flag:} \\
TARGET\_IS\_WINDOWS\_2000.

\IFRU{Во-вторых, файл}{Also}, dbghelp.dll \IFRU{из}{file from} Windows XP 
\IFRU{должен находится в той же директории что и}{should be located in the same
folder as} tracer.exe.

\chapter{\IFRU{Заключение}{Conclusion}}

\IFRU{Эта версия еще не была протестирована как следует. Так что будьте готовы к неожиданным падениям. Я очень рекомендую проводить все эксперименты в виртуальной машине.}{This release is not tested properly yet. So please be prepared for any possible crash. I strongly advice to do all experimentation in virtual machine.}

\IFRU{Если вы нашли ошибку, пожалуйста напишите мне:}{If you find any bug, please drop me a line:} \href{mailto:\EMAIL}{\EMAIL}. \IFRU{Пришлите также файл tracer.log и/или скриншот того что вывел tracer.}{Please attach tracer.log file and screenshot of the last tracer output.}

\IFRU{Я буду также благодарен любому комментарию или предложению насчет tracer.}{I'll also be thankful for any comments and suggestions related to tracer tool.}

\IFRU{Если вы чувствуете что ваш вклад в код стоит того чтобы быть включенным в мою версию, пожалуйста присылайте ваш патч.}{If you feel your contribution to source code is worth enough, please send me your patch.}

Tracer \IFRU{так же много используется для иллюстраций в моей книге 
``Краткое введение в reverse engineering для начинающих'', 
свободно доступной \href{http://yurichev.com/RE-book.html}{здесь}.}
{is also used a lot for illustration purposes in my ``Quick introduction to reverse engineering for beginners''
book, freely available \href{http://yurichev.com/RE-book.html}{here}.}

\end{document}
